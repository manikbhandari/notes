\documentclass{article}
\usepackage[final]{nips_2018}

\usepackage[utf8]{inputenc} % allow utf-8 input
\usepackage[T1]{fontenc}    % use 8-bit T1 fonts
\usepackage{hyperref}       % hyperlinks
\usepackage{url}            % simple URL typesetting
\usepackage{booktabs}       % professional-quality tables
\usepackage{amsfonts}       % blackboard math symbols
\usepackage{nicefrac}       % compact symbols for 1/2, etc.
\usepackage{microtype}      % microtypography
\usepackage{amsmath}
\usepackage{amsthm}
\usepackage{xcolor}
\usepackage{wrapfig}

\newtheorem{theorem}{Theorem}

\title{Game Theory}

\author{
	Manik Bhandari\\
	Department of Computational Data Science\\
	Indian Institute of Science\\
	Bangalore, India \\
	\texttt{manikb@iisc.ac.in} \\
}


\begin{document}
	
	\maketitle
	
	\begin{abstract}
		Notes of Game Theory mainly from lectures at IISC.
	\end{abstract}
	
\section{Introduction}
\textbf{Define} error of a classifier (aka \textit{true error}) as probability of it making a mistake given a random data point.
\[
    L_{D,f}(h) = \underset{x \sim D}P[h(x) \neq f(x)] = D(\{x: h(x) \neq f(x)\})
\]
where D is the distribution from where a data point x is drawn, $f$ is a known \textit{correct} function which always gives the correct labels to a data point. By this definition $D(A)$ is the probability of observing a random point $x$ from $A$.

\textbf{Define} \textit{training error} or \textit{emperical risk} as 
\[
    L_S(h) = \frac{|i \in [m]: h(x_i) \neq y_i|}{m}
\]
where $S$ is the training \textit{set} (it is actually a sequence since points can repeat and classifiers often take into account their order) of the form $\{(x_i, y_i)\}$. If you naively minimize this emperical risk then you are likely to overfit. To avoid it, you use some prior knowledge about the \textit{kind of classifier} that can possibly fit to the data and restrict your hypothesis search space to those types of classifiers.


This kind of restriction induces a \textit{bias} in the model (aka \textit{inductive bias}). In this setting, define 
\[
    h_S = ERM_h(S) \in \underset{h \in \mathcal{H}}{argmin}~L_S(h).
\]
This is a tradeoff -- choosing a restricted $\mathcal{H}$ can add too much bias but choosing a large $\mathcal{H}$ may lead to overfitting.

\paragraph{Finite hypothesis class} If we restrict $\mathcal{H}$ to have an upper bound on its size then $ERM_h$ will not overfit if we have \textit{large} training data (how large will depend on size of $\mathcal{H})$.

\paragraph{Realizability Assumption} There exists $h^* \in \mathcal{H}$ such that $L_{D,f}(h^*) = 0$ i.e. it never makes a mistake which means that $L_S(h^*) = 0$. Since this is the least possible error, this means that for every $ERM$ hypothesis $L_S(h_S) = 0$. We are however interested in true error of $h_S$ i.e. $L_{D,f}(h_S)$.

\paragraph{iid assumption} Assume that elements of $S$ are identically and independently distributed according to $D$ denoted by $S \sim D^m$.

Now, we would like to have an $h_S$ such that $L_{D,f}(h_S)$ is not too large. Let's say $h_S$ \textit{fails} if $L_{D,f}(h_S) > \epsilon$.

We want to upper bound the probability of sampling a training set that leads to a failure i.e $D^m({S: L_{D,f}(h_S) > \epsilon}).$ Define bad hypothesis as $\mathcal{H_B} = \{h \in \mathcal{H}\}: L_{D,f}(h_S) > \epsilon$ and misleading training sets as $M = \{S: \exists h \in \mathcal{H_B}, L_S(h) = 0\}$. So, all the training sets for which $h_S$ fails must be misleading (there can be other misleading sets also). So
\[
	\{S: L_{D,f}(h_S) > \epsilon \} \subseteq M = \bigcup_{h \in \mathcal{H_B}} \{S:  L_S(h) = 0\}.
\]
This means that 
\[
	D^m(\{S: L_{D,f}(h_S) > \epsilon \}) \leq D^m(M) = D^m(\bigcup_{h \in \mathcal{H_B}}\{S: L_S(h) = 0\}).
\]
Take union bound of RHS to get 
\[
	D^m(\{S: L_{D,f}(h_S) > \epsilon \}) \leq \sum_{h \in \mathcal{H_B}} D^m(\{S: L_S(h) = 0\}) = \sum_{h \in \mathcal{H_B}} \bigg(\prod_{i=1}^{m} D(\{x_i: h(x_i) = f(x_i)\})\bigg).
\]
and since $h \in \mathcal{H_B}$, $D(\{x_i: h(x_i) = f(x_i)\}) \leq 1 - \epsilon$. But each $x_i$ is iid over $D$ so $D^m(\{S: L_S(h) = 0 \leq (1-\epsilon)^m \leq e^{-\epsilon m}$. As $m$ goes large, the probability of finding a misleading set reduces. Therefore
\[
	D^m(\{S: L_{D,f}(h_S) > \epsilon \}) \leq |\mathcal{H_B}|e^{-\epsilon m} \leq |\mathcal{H}|e^{-\epsilon m}.
\]
Take log both sides to get
\[
\log D^m(\{S: L_{D,f}(h_S) > \epsilon \})  \leq \log |\mathcal{H}| - \epsilon m \implies m \leq \frac{\log( |\mathcal{H}|/\delta)}{\epsilon}
\]
where $\delta = D^m(\{S: L_{D,f}(h_S) > \epsilon \})$. This also implies that if $m$ is large enough i.e. $ m \geq \frac{\log( |\mathcal{H}|/\delta)}{\epsilon}$ then $L_{D,f}(h_S) \leq \epsilon$ with probability $1 - \delta$ of choosing the iid samples $S$. \\

So with the $ERM_h$ rule, your hypothesis will be \textit{probably} ($1-\delta$) \textit{approximately} ($\epsilon$) \textit{correct} (PAC). Note that the size $m$ does not depend upon the underlying distribution or labeling function.

\paragraph{PAC Learnability} A hypothesis class $\mathcal{H}$ is PAC learnable if $\exists~ m_{\mathcal{H}}: (0,1)^2 \to \mathbb{N}$ and a learning algorithm such that\\
For every $(\epsilon, \delta) \in (0,1)$, for every distribution $\mathcal{D}$ over $\mathcal{X}$ and for every labeling function $f: \mathcal{X} \to (0,1)$\\
If the realizability assumption holds over $\mathcal{H}, \mathcal{D}, f$\\
then running the algorithm on $m > m_{\mathcal{H}}(\epsilon, \delta)$ samples generated iid from $\mathcal{D}$ and labeled by $f$ gives a hypothesis $h$ such that \\
with probability at least $1-\delta$ over the choice of examples, $L_{\mathcal{D}, f}(h) \leq \epsilon$.

\paragraph{Sample complexity} $m_{\mathcal{H}}: (0,1)^2 \to \mathbb{N}$ defines the \textit{sample complexity} of learning $\mathcal{H}$ i.e. how many samples are required to get a PAC solution. Let it be the \textit{minimum function} that satisfies the criteria of PAC learnability.

\paragraph{Sample complexity of finite hypothesis class} Every finite hypothesis class is PAC learnable with sample complexity $ m \leq \lceil \frac{\log( |\mathcal{H}|/\delta)}{\epsilon} \rceil$

\paragraph{Removing realizability assumption} Assuming that such an $h^*$ exists such that \\
$\underset{x \sim \mathcal{D}}{P}[h^*(x) = f(x)] = 1$ is too strong. Not only might such an $h^*$ not exists, your features might not be discriminative enough. Instead assume that $\mathcal{D}$ is a joint distribution over domain points $\mathcal{X}$ and labels $\mathcal{Y}$.
Now, true error 
\[
	L_D(h) = \underset{(x,y~ \sim ~\mathcal{D})}{P}[h(x) \neq y] = \mathcal{D}(\{(x,y): h(x) \neq y\})
\]

\paragraph{Bayes optimal predictor} is the best labeling function defined as
\[
	f_{\mathcal{D}}(x) = 
        \begin{cases} 
            \text{1} \quad \text{if} ~\mathbb{P}[y=1 | x] \geq 0.5\\
            \text{0} \quad \text{otherwise}
        \end{cases}
\]
i.e. there is no other labeling function with a lower true error rate.

\paragraph{Agnostic PAC Learnability} A hypothesis class $\mathcal{H}$ is agnostic PAC learnable w.r.t. a set $\mathbb{Z}$ and a loss function $l: \mathcal{Z} \to \mathbb{R}_+$ if there exists a function $m_\mathcal{H}: (0,1)^2 \to \mathbb{N}$ and a learning algorithm such that\\
for \textit{every} $\epsilon, \delta \in (0,1)$ and for  \textit{every} $\mathcal{D}$ over $\mathcal{Z}$ when the algorithm is run $m \geq m_\mathcal{H}(\epsilon, \delta)$ samples iid from $\mathcal{D}$, the algo returns a hypothesis $h \in \mathcal{H}$ such that with probability $1 - \delta$ over the training samples
\[
	L_\mathcal{D}(h) = \underset{h' \in \mathcal{H}}{min} L_\mathcal{D}(h') + \epsilon
\]
where $L_\mathcal{D}(h) = \mathbb{E}_{z\sim\mathcal{D}}[l(h,z)]$.
\section{Utility Theory}
\paragraph{Every game of chess ends in either a win for white, win for black, or draw.}
Simple proof idea: think of the game of chess as a tree with children being the reachable board positions from parent. Start from the leaf and build up.

\paragraph{preference relation} A player wants his \textit{best possible outcome}. To measure this we want a relation between all possible outcomes. \textcolor{red}{An order?}. This relation is \textit{preference relation}.
It is a binary relation over the set of possible outcomes $O$ i.e. $x \geq_{i} y$ implies that player $i$ prefers outcome $x$ over $y$ or is indifferent between the two.

\paragraph{Assumptions about the preference relation}
\begin{enumerate}
    \item It is complete i.e. for any pair of outcomes $x, y$ in $O$ either $x \geq_{i} y$ or $x \leq_{i} y$  or both.
    \item It is reflexive i.e. $x \geq_{i} x$ for all $x \in O$.
    \item It is transitive i.e. for any $x,y,z \in O$ if $x \geq_{i} y$ and $y \geq_{i} z$ then $x \geq_{i} z$.
\end{enumerate}

\paragraph{Utility function} is a mapping from $O \to \mathbb{R}$ such that if $x \geq_{i} y$ then $u(x) \geq_{i} u(y)$. Essentially it is an association of preference with a real number.
Note that for any monotonically increasing function $v:\mathbb{R} \to \mathbb{R}$, $v \cdot u = v(u(x))$ is also a utility function for the same preference relation. So a utility function $u$ is only \textit{ordinal} i.e the real number is not a measure of intensity of a player's preference.

\section{Strategic form games a.k.a. Normal form}
\paragraph{Definition} A Strategic Form Game is an ordered triple $G = (N, (S_i)_{i \in N}, (u_i)_{i \in N})$. $N$ is the \textit{finite} set of players $N = {1,2,..,n}$, $S_i$ is the set of strategies of player $i$. Denote the set of all possible strategy vectors as $S = S_1 \times S_2 ... \times S_n$. And $u_i: S \to \mathbb{R}$ maps a vector of strategies $s = (s_i)_{i\inN}$ to the payoff of the $i^{th}$ player. 

Note the strategy set of a player need not be finite. If it is finite, we call it a \textit{finite game}.

\paragraph{Notation} Denote $X$ as the cross product $\times_{i\inN}X_i$. Then $X_{-i} = $\times_{j\neq i}X_j$ i.e. an element of $X_{-i}$ is an n-1 dimensional vector $x_{i-1} = (x_i, ... x_{i-1}, x_{i+1}... x_n)$.

\paragraph{Domination} The strategy $s_i$ of a player is dominated by $t_i$ if for \textit{each} strategy of other players $s_{-i} \in S_{-i}$ 
\[
    u_i(s_i, s_{-i}) < u_i(t_i, s_{-i})
\]
So $t_i$ strictly dominates $s_i$ or $s_i$ is strictly dominated by $t_i$.

\paragraph{Assumptions} A rational player will never choose a strictly dominated strategy and all players in the game are rational. Also assume rationality is common knowledge.

\paragraph{Strictly dominant strategy} of a player is one which strictly dominates all other strategies of the player.

\section{Strategic form games a.k.a. Normal form}
\paragraph{Definition} A Strategic Form Game is an ordered triple $G = (N, (S_i)_{i \in N}, (u_i)_{i \in N})$. $N$ is the \textit{finite} set of players $N = {1,2,..,n}$, $S_i$ is the set of strategies of player $i$. Denote the set of all possible strategy vectors as $S = S_1 \times S_2 ... \times S_n$. And $u_i: S \to \mathbb{R}$ maps a vector of strategies $s = (s_i)_{i \in N}$ to the payoff of the $i^{th}$ player. 

Note the strategy set of a player need not be finite. If it is finite, we call it a \textit{finite game}.

\paragraph{Notation} Denote $X$ as the cross product $\times_{i \in N}X_i$. Then $X_{-i} = \times_{j\neq i}X_j$ i.e. an element of $X_{-i}$ is an n-1 dimensional vector $x_{i-1} = (x_i, ... x_{i-1}, x_{i+1}... x_n)$.

\paragraph{Domination} The strategy $s_i$ of a player is dominated by $t_i$ if for \textit{each} strategy of other players $s_{-i} \in S_{-i}$ 
\[
    u_i(s_i, s_{-i}) < u_i(t_i, s_{-i})
\]
So $t_i$ strictly dominates $s_i$ or $s_i$ is strictly dominated by $t_i$.

\paragraph{Assumptions} A rational player will never choose a strictly dominated strategy and all players in the game are rational. Also assume rationality is common knowledge.

\paragraph{Strictly dominant strategy} of a player is one which strictly dominates all other strategies of the player.

\paragraph{Iterated elimination of dominant strategies} Since all players are rational and this is common knowledge, and rational players never choose a dominated strategy, we can safely eliminate such a strategy from a player's strategy set. Iteratively doing this is called Iterated elimination of dominated strategies. When this process yields only \textit{one} strategy per player, the strategy vector obtained is called \textit{solution} of game.

Therefore, solution always exists if each player has a \textit{strictly dominant strategy}.

\paragraph{Order of elimination} does not matter when we iteratively eliminate strictly dominated strategies. You'll always get the same set of strategies. \textcolor{red}{prove this}. However, for eliminating weakly dominated strategies, you might get the different results.

\paragraph{Weakly dominated strategy} $s_i$ is weakly dominated by $t_i$ if 
\begin{enumerate}
	\item for every strategy $s_{-i} \in S_{-i}$ of other players, $u_i(s_i, s_{-i}) \leq u_i(t_i, s_{-i})$ \textit{and}
	\item there exists at least one strategy $t_{-i} \in S_{-i}$ of other players such that $u_i(s_i, t_{-i}) < u_i(t_i, t_{-i})$.
\end{enumerate}

Henceforth, dominated = weakly dominated. A rational player will never choose a dominated strategy. A \textit{rational strategy} is one which is reached by iterative elimination of weakly dominated strategies a.k.a. \textit{rationalizability}. 

Elimination of a player's strictly dominated strategy does not require the knowledge of other players' payoffs. However, in rationalizability when we eliminate $s_i$ of player $i$ after $s_j$ of player $j$, player $i$ assume that player $j$ will never play $s_j$.

\paragraph{Second price auction as Strategic Form Game} Let $N$ be the set of players, $S_i = [0, \infty]$ is the set of bids that the player can make, utility for a given strategy vector $b$ is 
\[
u_i(b) = 
\begin{cases}
0 \quad \quad \quad \quad \quad \quad \quad \text{if} \quad b_i \neq \underset{j}{max}b_j\\
\frac{v_i - \underset{i \neq j}{max}b_j}{|\{k: b_k = \underset{j}{max}b_j\}|} \quad \quad \text{if} \quad b_i = \underset{j}{max}b_j
\end{cases}
\]
Second case arises from the fact that $k$ people can give the same maximum bid in which case, the winner will be decided by a fair lottery. Note that, in this case the second highest bid will also be the same as $v_i$ and so the profit will be 0.

In this game, the strategy $b_i = v_i$ weakly dominates all other strategies. \textcolor{red}{Verify by drawing the utility functions}.

\paragraph{Three Definitions of Nash Equilibrium} 
\begin{enumerate}
	\item A strategy vector $s^* = (s_1^*, s_2^*, ..., s_n^*)$ is a Nash Equilibrium if for \textit{each} player $i \in N$ and for each strategy $s_i \in S_i$, $u_i(s*) \geq u_i(s_i, s_{-i}^*)$. And the corresponding payoff vector $u(s^*)$ is the \textit{equilibrium payoff }corresponding to the Nash eqbm $s^*$.
	\item \textbf{Profitable Deviation} of player $i$ from a strategy vector $s$, is a strategy $\hat{s_i} \in S_i$ such that $u_i(\hat{s_i}, s_{-i}) > u_i(s_i)$. A Nash eqbm is a strategy vector from which no player has a profitable (unilateral) deviation.
	\item \textbf{Best Response} $s_i$ is a \textit{best response} of player $i$ to $s_{-i}$ if $u_i(s_i, s_{-i}) \geq \underset{t_i \in S_i}{max}~u_i(t_i, s_{-i})$. $s^* = (s_1^*, s_2^*, ...)$ is a Nash Equilibrium if $s_i^*$ is a best response to $s_{-i}^*$ for every player $i \in N$.
\end{enumerate}

\paragraph{Coordination Games} are those in which coordinating responses leads to better payoffs than not coordinating.

\paragraph{Properties of Nash Equilibrium}
\begin{enumerate}
	\item  It is stable - a property that every equilibrium should exhibit. 
	\item  It is self-fulfilling - If there is a agreement to play \textit{some} equilibrium, then even if the agreement was not binding, players would prefer not to deviate.
	\item It is also a sort of prescription - a judge would prescribe such an equilibrium to players so that they don't deviate. 
	\item  It can also be interpreted as a prediction - what \textit{would} happen if rational players with common knowledge play this game?
\end{enumerate}

\section{Logistic Regression}
Used for binary classification and equivalent to 1 layer Neural Network \textcolor{red}{show how}. Let set of classes = $\{0,1\}$ and training set be $S = {(x^1, y^1), (x^2, y^2), ... (x^m, y^m)}$ where each $x^i \in \mathbb{R}^d$ and $y^i \in \{0,1\}$. Classification function will predict the probability of label being 1 (or 0 since their sum is 1). So, we'll have
\[
	p_m(y=0|x) + p_m(y=1|x) = 1.
\]
To make predictions using this, we'll use
\[
y = 
\begin{cases} 
\text{1} \quad \text{if} ~p_m[y=1 | x] > 0.5\\
\text{0} \quad \text{otherwise}
\end{cases}
\]
The function to model probabilities will be 
\[
	p_{w,b}(x) = \sigma(<w,x> +~ b)
\]
where $W \in \mathbb{R}^d$ and $b \in \mathbb{R}$. To get a probability, we'll pass the linear model through a \textit{sigmoid} function i.e. $\sigma(t) = \frac{1}{1 + e^{-t}} = 1-\sigma(-t)$.

If we define our hypothesis $h$ as above i.e. 
\[
h_{w,b}(x) = 
\begin{cases} 
\text{1} \quad \text{if} ~p_{w,b}[x] > 0.5\\
\text{0} \quad \text{otherwise}
\end{cases}
\]
then the set $\mathcal{H} = \{h_w,b: w\in\mathbb{R}^d, b\in\mathbb{R}\}$ defines the \textit{half spaces} hypothesis class, which we know has a VC-dimension of $d+1$. \textcolor{red}{prove this}.

Now, since $L_\mathcal{D}(\mathcal{H}) \leq L_S(\mathcal{H}) + O\big(\sqrt{\frac{VC(\mathcal{H}) + log(1/\delta)}{m}}\big)$ for $m>>d$ ERM should give a small generalization error.

\paragraph{Solving it} We need to find $\underset{w,b}{\text{argmin}}L_S(h_{w,b})$ which is
\[
	L_S(h_{w,b}) = \frac{1}{m}\sum_{i}L(y^i \neq h_{w,b}(x^i))
\] 
such that for \textit{all $i$}
\[
	wx^i + b > 0 \quad \text{if} \quad y^i = 1 \quad \text{and} \quad
	wx^i + b < 0 \quad \text{if} \quad y^i = 0,	
\]

This is a linear program which is solvable in polynomial time in $m,d$ if realizable. If not, then finding the argmin is NP-hard. It is also not obvious how to find an approximate solution to this expression.

You can try converting this into a \textit{least-squares} problem i.e. 
\[
	\underset{w,b}{\text{min}} \sum_{i \in[m]}(<w,x^i> + b - y^i)^2
\]
There is an efficient way to solve this but this is solving the wrong problem! We don't want to value of $<w,x^i> + b$ to be exactly $y^i$.

\paragraph{MLE Approach} Another approach may be to maximize the MLE i.e. 
\[
	\text{Likelihood(s)} = \prod_{i \in [m]}p_{w,b}(y^i|x^i)
\]
where $p_{w,b}(y=1|x) = \sigma(<w,x> + b)$ and $p_{w,b}(y=0|x) = 1 - \sigma(<w,x> + b)$.
This implies that 
\[
	p_{w,b}(y|x) = \hat{y}^y(1-\hat{y})^{(1-y)} \implies log(p_{w,b}(y|x)) = ylog(\hat{y}) + {(1-y)}log(1-\hat{y})
\]
So we need to minimize the \textit{cross entropy loss}
\[
	L_{CE}(w,b;S) = -\bigg(\sum_{i \in [m]}y^ilog(\sigma(<w,x^i> + b)) + (1 - y^i)log(1 - \sigma(<w,x^i> + b))\bigg)
\]

This minima might not always exist. Let's say we keep $y = 1$, then the first term can be taken to $-\infty$ by making the weights arbitrarily large.


\section{Multinomial Logistic Regression}
Let set of classes = $\{1,2...k\}$ and training set be $S = {(x^1, y^1), (x^2, y^2), ... (x^m, y^m)}$ where each $x^i \in \mathbb{R}^d$ and $y^i \in [k]$. 

There are multiple ways to model this, one way is to use \textit{one vs all} approach. The other is to use a logistic regression approach. Here, we take the latter. \textcolor{red}{Work out the expression for one vs all case.} Use $k$ linear functions as 
\[
	<w_1,x> + b_1, ..., <w_k, x> + b_k.
\]
We'll also make use of the softmax function which takes a vector of inputs $(t_1, t_2...)$ and transforms them to $(\frac{e^{t_1}}{\sum e^{t_i}}, ... )$. This has the nice property that its sum to one. \textcolor{red}{add imp props of softmax}.

We then use the \textit{Maximum Likelihood Approach}. Define Likelihood as
\[
	\prod_{i=1}^{m}p(y^i|x^i) = p(1|x)^{\mathbbm{1}_(y=1)}...p(k|x)^{\mathbbm{1}_(y=k)}
\]

Maximizing this equivalent to minimizing the \textit{Cross Entropy Loss} \textcolor{red}{Why is it called cross entropy}.
\[
	L_{CE}(w,b;S) = -\sum_{i\in[k]}\mathbbm{1}_(y=1)log\bigg(\frac{e^{<w_i,x>+b_i}}{\sum e^{<w_i,x>+b_i}}\bigg) = log\bigg( \frac{e^{<w_{y^i},x>+b_{y^i}}}{\sum e^{<w_{y^i},x>+b_{y^i}}} \bigg)
\]

\section{Feed forward Neural Networks}
Assume 2 layers - hidden layer and output layer. We never count input layer. 

Let $f = (f_1, ..., f_k)$ such that $f:\mathbb{R}^d \to \mathbb{R}^k$.

Let $f_i(x) = \sum_{r\in[m]}a_{ir}\rho(<w_r,x> + b_r)$ where $w_r \in \mathbb{R}^d, a_{ir}, b_r \in \mathbb{R}$ and $\rho:\mathbb{R}\to\mathbb{R}$ is the \textit{activation function}. \textcolor{red}{Note different types of activation functions and their properties. When to use which.}

This can be written in a compact form as $f(x) = A\rho(Wx + b)$ where each row of $A$ is $a_i$ and each row of $W$ is $w_r$.

Note that, $A$ takes the role of $W$ in the next layer. So, to generalize to further layers, you'll find:

For $l$ layers: $f(x) = \rho(W^l(...\rho(W^2\rho(W^1x))))$.

\paragraph{Survey of techniques}
\begin{enumerate}
	\item \hyperlink{https://people.csail.mit.edu/rivest/pubs/BR93.pdf}{[Blum and Rivest, 1992]} n input units, 2 hidden units, 1 output unit, $|S|=O(d)$ where $x\in\mathbb{R}^d$, Activation function $\rho$ is \textit{threshold function}. It's NP-Hard to fit this network to an $S$.
	\item \hyperlink{https://ac.els-cdn.com/S0304397501000573/1-s2.0-S0304397501000573-main.pdf?_tid=e2ae69f9-606c-4b53-a085-e8919afbe4f1&acdnat=1548162465_310e08a262614701ef11c40c8999c1c6}{[Bartlett and Ben-David]} Similar architecture as above except $k$ hidden units. For a realizable $S$, finding weights that fit at least $(1 - \frac{c}{k})$ fraction of examples is NP-hard, where c is some constant.
	\item \hyperlink{https://www.mitpressjournals.org/doi/pdf/10.1162/089976602760408035}{[Sima 2002]} One neural, sigmoid activation, NP-hard to train.
	\item \hyperlink{}{[BLR 18, MR18]} Same results as above but for ReLU networks.
	\item \hyperlink{}{[Jones, VN 1998]} 1 Hidden layer of sigmoid with units polynomial in $d$. Output linear with non-negativity. Also NP-hard.
	
\end{enumerate}
\section{Solutions other than Nash Equilibrium}
We will find solutions that \textit{rational} players might not choose but \textit{pessimistic} players might.

A pessimistic player might want to maximize the minimum payoff he can get, disregarding whether other players are rational or not. i.e. 
\[
	\underline{v_i} := \underset{s_i \in S_i}{max}~\underset{t_{-i} \in S_{-i}}{min}~u_i(s_i, t_{-i})
\]
This quantity $\underline{v_i}$ is the \textit{maxmin value} of player $i$ or the \textit{security} of player $i$. The strategy $s_i^*$ that guarantees this value is the \textit{maxmin strategy}.
	
	
\end{document}